% %
% LAYOUT_E.TEX - Short description of REFMAN.CLS
%                                       99-03-20
%
%  Updated for REFMAN.CLS (LaTeX2e)
%
\documentclass[twoside,a4paper]{refart}
\usepackage{makeidx}
\usepackage{ifthen}
% ifthen wird vom Bild von N.Beebe gebraucht!

\def\bs{\char'134 } % backslash in \tt font.
\newcommand{\ie}{i.\,e.,}
\newcommand{\eg}{e.\,g..}
\DeclareRobustCommand\cs[1]{\texttt{\char`\\#1}}

\title{WARP User's Guide}
\author{Ryan M. Bergmann \thanks{ryanmbergmann@gmail.com} \\
Kelly L. Rowland \thanks{krowland@berkeley.edu}\\
v 1.0,  May 2015}

\date{}
\emergencystretch1em  %

\pagestyle{myfootings}
\markboth{WARP User's Guide}%
                {WARP User's Guide}

\makeindex 

\setcounter{tocdepth}{2}

\begin{document}

\maketitle

%\begin{abstract}
%
%This is the user guide for WARP.
%
%\end{abstract}

This user's guide is meant to introduce a user to the way in which WARP executes and how it's top-level functionality can be used to solve neutron transport problems.  For more detailed information about the algorithms WARP uses, please refer to article ``Algorithmic choices in WARP - A framework for continuous energy
Monte Carlo neutron transport in general 3D geometries on GPU'' in Annals of Nuclear Engineering (doi:10.1016/j.anucene.2014.10.039).

\tableofcontents

\newpage


%%%%%%%%%%%%%%%%%%%%%%%%%%%%%%%%%%%%%%%%%%%%%%%%%%%%%%%%%%%%%%%%%%%%

\section{An Introduction to WARP}

WARP, which XXXXXXXX, is a C\texttt{++} library that contains a set of classes and data structures that allow Monte Carlo neutron transport to be performed using graphics processing units (GPUs).  It is written to scale well and execute efficiently on GPUs.  Typical CPU codes track a single neutron life, from cradle to grave, marking its interactions along the way.  WARP does things a little differently.  WARP applies the main transport loop to an entire (preferably large) set of neutrons instead of a single neutron.  The parallelism in WARP comes from acting on many pieces of data concurrently instead of executing many tasks concurrently.

FIGURE OF THE DATA PARALLELISM UNDER THE TRANSPORT LOOP

\section{Compiling WARP}

WARP has the following dependencies:
\begin{itemize}
	\item{C\texttt{++} compiler}
	\item{CMake ($>=$ 2.8.10)}
	\item{NVIDIA CUDA ($>=$ 5.0)}
	\item{NVIDIA OptiX ($>= 3.0.1$)}
	\item{CUDPP ($>=$ 2.1)}
	\item{PyNE}
	\item{SWIG}
	\item{GoogleTest}
\end{itemize}

WARP is compiled with CMake.

\begin{verbatim}
   \documentclass[11pt,twoside,a4paper]{article}
   \usepackage{mysty} %<- This calls the package "mysty"
\end{verbatim}


\section{The Library Interface}

WARP has two interfaces available for use, one in C\texttt{++} and the other in Python.

\subsection{C\texttt{++}}

\subsection{Python}

\begin{description}

\item[Line spacing]\index{line spacing}
        The spacing between two lines should be larger than the spacing 
        between two words to guide the eyes of the reader.

\item[Line length]\index{line length}
        The length of a line -- or when using multicolumn layout of a 
        column -- should be about 60 characters. When lines get longer they 
        are more difficult to read and it is easier to go to the wrong line 
        after finishing the current one. Increasing the linespacing may help a 
        little.
        When lines get to short it is difficult to set them justified, and you 
        will get lots of hyphenated words.
        
\item[Page layout]\index{page layout}
        Normal text pages should look the same throughout the document. 
        Figures, tables and special pages like the index need not appear in 
        the same layout but should take as much space as needed.
        
\item[Margin notes]\index{margin notes}
        Margin notes are often more suitable than footnotes because they 
        appear right next to the text they refer to. Special margin notes are 
        the ``attention sign'' or the ``dangerous bend'' that guide the user 
        to important parts of the text.
        
\item[Headings and Footings]\index{headings}\index{footings}
        Headings and footings should make it easier for the reader to orient
        himself in the document. If you expect readers to copy single pages
        from the document they should contain information about the paper as
        a whole, just in case you need more information or want to cite the
        whole paper.
        
        If you expect the document to change often (like software manuals),
        each page should contain a version information or at least a date.
        
\end{description}

\section{The Art of Layout-Design}
\label{design}
\index{layout design}

\subsection{Common Rules}
\label{design}
\index{rules}



\subsection{Special note for technical descriptions}



%%%%%%%%%% %%%%%%%%%%%%%% %%%%%%%%%%%% %%%%%%%%%%%% %%%%%%%% %%%%%%%%%%%%

%%%%%%%%%%%%%%%%%%%%%%%%%%%%%%%%%%%%%%%%%%%%%%%%%%%%%%%%%%%%%%%%%%%%%%

%\input content

%%%%%%%%%%%%%%%%%%%%%%%%%%%%%%%%%%%%%%%%%%%%%%%%%%%%%%%%%%%%%%%%%%%%%%

\printindex

\end{document}
